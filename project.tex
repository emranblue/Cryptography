\documentclass{report}
\usepackage[margin=1.3 in]{geometry}
\usepackage{graphicx}
\begin{document}
\tableofcontents
\chapter{ Cryptography }
\section{ What is cryptography}

%-------defination----------
\Large{ Cryptography is the science of secret communications.
It is the study of secure communications techniques that allow only the sender and intended recipient of a message to view its contents. The term is derived from the Greek word kryptos, which means hidden. It is closely associated to encryption, which is the act of scrambling ordinary text into what's known as ciphertext and then back again upon arrival. In addition, cryptography also covers the obfuscation of information in images using techniques such as microdots or merging. Ancient Egyptians were known to use these methods in complex hieroglyphics, and Roman Emperor Julius Caesar is credited with using one of the first modern ciphers.


Long before the information age, cryptography was used only to ensure secrecy of information. Encryption was used to ensure confidentiality in communications by spies, military leaders and diplomats. The Egyptian hieroglyphs, the scytale transposition cipher used by the Spartans of Greece, waxed seals and different physical devices to assist with ciphers were used throughout history right up to modern times. These devices underwent further changes when computers and electronics came into the picture, immensely helping in cryptanalysis.Cryptography has become more mathematical now and also finds applications in day-to-day security. It helps you safely transfer or withdraw money electronically and you’d be hard-pressed to come across an individual without a credit or debit card. The public-key encryption system introduced the concept of digital signatures and electronic credentials. Cryptography has a definitive existence in our lives today and the whole system will crumble in its absence. Let’s now discuss the varied uses of cryptography in modern times and its intersection with computer science.
\begin{center}
\includegraphics[scale=0.4]{"crypto.png"}
\end{center}

%---------usage------------------
\section{Usage of cryptography}
\subsection*{Secrecy in transmission}
\Large{The major goal of cryptography is to prevent data from being read by any third party. Most transmission systems use a private-key cryptosystem. This system uses a secret key to encrypt and decrypt data which is shared between the sender and receiver. The private keys are distributed and destroyed periodically. One must secure the key from unauthorized access, because any party that has the key can decrypt the encrypted information.}\\

\begin{center}
\includegraphics[scale=0.6]{"public_private.png"}
\end{center}

\Large{Alternately a key-generating-key, called a master key, can be used to electronically generate a one-time session-key for every transaction. The secrecy of the master-key should be maintained by all parties privy to the information. The disadvantage of this method is there’s too much hope riding on the master-key, which if cracked, collapses the entire system.

 

A better method is to use a public-key cryptosystem. In this system, data can be encrypted by anyone with the public-key, but it can be decrypted only by using the private-key, and data that is signed with the private key can be verified only with the public key. With the development of publickey systems, secrecy can be maintained without having to keep track of a large number of keys or sharing a common master-key. If, say, Alex wants to communicate with Neil, she first generates her public/private key pair and sends the public key to Neil over a non-secure channel. Neil  to encrypt information and sends it back to Alex. Only Alex has the private key with which she can decrypt the information. Anyone who intercepts the public key or the encrypted data can’t decrypt the message due to the protocols followed during information transfer.}


\subsection*{Secrecy in storage}

\Large{Storage encryption refers to the application of cryptographic techniques on data, both during transit and while on storage media. Storage encryption is gaining popularity among enterprises that use storage area networks (SANs). Secrecy in storage is maintained by storing data in encrypted form. The user has to provide the key to the computer only at the beginning of a session to access the data and it then takes care of encryption and decryption throughout the course of normal use. Hardware devices can also be used for PCs to automatically encrypt all information stored on disk. When the computer is turned on, the user must supply a key to the encryption hardware. The information is plain gibberish without its key thus preventing misuse if the disk is stolen.}

\begin{center}
\includegraphics[scale=0.6]{"file_encryption.png"}
\end{center}

\Large{Multiple ciphers can be used for individual files and folders. The ciphers and keys should be changed frequently to ensure security of data. However, if the user forgets a key, all of the information encrypted with it makes no sense and is rendered useless. This is why backups of encrypted information are advised to be stored in plaintext. The data is only encrypted while in storage, not when in use. This leaves a loophole for the attackers. The system is vulnerable to a security breach if the encryption and decryption are done in software, or if the key is stored somewhere in the system.



\subsection*{Integrity in transmission}
\Large{We can use cryptography to provide a means to ensure that data is not altered during transmission, i.e. its integrity is preserved. In electronic funds transfer, it is very important that integrity be maintained. A bank can lose millions if a transaction is illicitly intercepted. Cryptographic techniques are employed to prevent accidental or intentional modification of data during transmission, leading to erroneous actions. One of the ways to ensure integrity is to perform a checksum on the information being transmitted and to transmit the checksum in an encrypted form as well.

The information is received on the other end and again checksummed. The transmitted checksum is decrypted and compared with the previous checksum. If the checksums agree, the information is most likely unaltered. The problem with this scheme is that the checksum of the original message can be known and another message with the same checksum can be generated and sent instead of the original one. This problem can be overcome by using a public-key cryptosystem. After generating the public-key/private-key pair, if we throw away the private-key and use only the public-key to encrypt the checksum, the checksum becomes impossible to decrypt. In order to verify the checksum, we generate a new checksum for the received information, encrypt it using the public-key and match it with the encrypted checksum. This is also known as a one-way function as it is hard to invert.}

\subsection*{Integrity in storage}

\Large{Integrity in storage had been ensured by access control systems with lock and keys and other guards to prevent unauthorized access to stored data. The existence of computer viruses has changed the scenario and the need of integrity against intentional attack has become a problem of epic proportions. Cryptographic checksums to ascertain validity of stored data are of help here. As in the case of transmission, a cryptographic checksum is produced and compared to the expected value. However, storage media are more vulnerable to attacks than transmission channels due to longer exposure and larger volumes of information.}


\subsection*{Authentication of identity}
\Large{Authentication is the process of verifying if the user has enough authority for data access. Simple passwords are used to identify someone. You must also have seen in classic gangster movies, the exchange of keywords to prove identity. Cryptography is similar to the practice of providing passwords for identity authentication. Modern systems use cryptographic transforms in conjunction with other characteristics of individuals to provide more reliable and efficient authentication of identity. Many systems allow passwords to be stored in an encrypted form, with read access available to all programs which may use them. Since passwords are not stored as plaintext, an accidental of data doesn’t compromise the system’s security.Passwords are analogous to the key in a cryptosystem that allows encryption and decryption of anything the password has access to. The principal element of this system is the password selection process. And that’s a whole other subject that we can’t cover here. But in a nutshell, the longer the password, the more random it will be and the harder it is to guess. So if you think it’s easy for you to remember, you should know that it will be all the easier to crack.}

\subsection*{Digital signatures}

\Large{A digital signature is a mechanism by which a message is authenticated i.e. proving that a message is coming from a given sender, much like a signature on a paper document. To be as effective as a signature on paper, digital signatures must be hard to forge and accepted in a court of law as binding upon all parties to the transaction. The need for digital signatures arises when the parties dealing in a transaction are not physically close, and the volume of paperwork is high, in other words big business dealings. Digital signatures can be created using a public key cryptosystem and hashing process. }\\
\begin{center}
\includegraphics[scale=0.6]{"digital_sig.png"}
\end{center}

\Large{Hashing produces a message digest that is a small and unique representation of the original message. Hashing is a one-way algorithm, i.e. the message can’t be derived from the digest. Let’s say that Alex is sending a message to Neil. Alex first hashes the message to produce a digest, and then encrypts the digest with her private-key to create her personal signature; the public-key and hash algorithm are appended to it. The whole message including the digest is then encrypted using a one-time symmetric-key which is known only to Alex and Neil. Neil decrypts the message using the symmetric-key. He then decrypts the message digest using the public-key. He would then hash the original message using the same hash algorithm (whose name was appended in the message) with which it was previously hashed. If the evaluated digest and decrypted digest match, then the signature has been verified and the recipient would be sure that the message integrity has been preserved.

Another aspect of this system is the non-repudiation of digital signatures. Since the private-key is only privy to the sender, he can’t deny signing the message. Also, a digital signature can be verified by anyone using the sender’s public-key which is usually included in the digital signature format.}
%----type----
\section{Types of cryptography}
In general there are three types Of cryptography:
\subsection{Symmetric Key Cryptography:}
\Large{It is an encryption system where the sender and receiver of message use a single common key to encrypt and decrypt messages. Symmetric Key Systems are faster and simpler but the problem is that sender and receiver have to somehow exchange key in a secure manner. The most popular symmetric key cryptography system is Data Encryption System(DES).}
\subsection{Hash Functions:}
\Large{There is no usage of any key in this algorithm. A hash value with fixed length is calculated as per the plain text which makes it impossible for contents of plain text to be recovered. Many operating systems use hash functions to encrypt passwords.}

\subsection{Asymmetric Key Cryptography:}
\Large{Under this system a pair of keys is used to encrypt and decrypt information. A public key is used for encryption and a private key is used for decryption. Public key and Private Key are different. Even if the public key is known by everyone the intended receiver can only decode it because he alone knows the private key.}


\section{History of cryptography}
\Large{Cryptology was well established in ancient times, with both Greeks and
Romans practicing different forms of cryptography. With the fall of the Roman
Empire, cryptology was largely lost in the West until the Renaissance, but it flour-
ished in the Arabic world. The Arabs invented the first reliable tool for cryptanaly-
sis, frequency analysis. With the end of the Middle Ages and the increase in
commerce and diplomacy, cryptology enjoyed a Renaissance of it’s own in the
West


Julius Caesar, probably the greatest of all Roman generals, was no stranger to cryp-
tology. In his famous Commentary on the Gallic Wars, Caesar himself describes
using a form of a cipher to hide a message If he had anything confidential to say, he wrote it in cipher, that is, by so changing the order of the
letters of the alphabet, that not a word could be made out. If anyone wishes to deci-
pher these, and get at their meaning, he must substitute the fourth letter of the alpha-
bet, namely D, for A, and so with the others.” (Seutonius 1957, Ch. 56) This is the
first written description of the modern monoalphabetic substitution cipher using a
shifted standard alphabet. Using Caesar’s cipher, the cipher alphabet looks like}

\begin{center}
\includegraphics[scale=0.3]{"ceser.png"}
\end{center}

\Large{For 900 years the monoalphabetic substitution cipher was the strongest cipher sys-
tem in the Western world. The Romans used it regularly to protect their far-flung
lines of communication. But after the fall of the Western Roman Empire in 476 C.E.
the knowledge of cryptology vanished from the West and wasn’t to return until the
Renaissance. Indeed, with the decline of literacy and scholarship in Europe during
the Dark Ages following the fall of Rome cryptology turned from a useful technique
for keeping communications secret into a dark art that bordered on magic.
But interest in cryptology was not dead. In the latter part of the first millennium,
there was another place where intellectual curiosity and scholarship flowered and
where mathematics and cryptology saw their biggest advances since Caesar – the
Arab world. And it was the Arab world from which the next big advance in crypt-
analytic techniques would come.
The period around the ninth century C.E. is considered to be the beginning of the
Islamic Golden Age, when philosophy, science, literature, mathematics, and reli-
gious studies all flourished in what was then the peace and prosperity of the Abbasid
Caliphate. Into this period was born Abu Yūsuf Ya-qūb ibn Isāq as-Sabbāh al-Kindi
(801–873 C.E.), a polymath who was the philosopher of the age. Al-Kindi wrote
books in many disciplines including astronomy, optics, philosophy, mathematics,
medicine, and linguistics, but his book on secret messages for court secretaries, A
Manuscript on Deciphering Cryptographic Messages is the most important to the
history of cryptology. It is in this book that the technique of frequency analysis is
first described.}

\begin{center}
\includegraphics[scale=0.6]{"al_kindi.png"}
\end{center}


\Large{Although cryptography has a long and complex history, it wasn't until the 19th century that it developed anything more than ad hoc approaches to either encryption or cryptanalysis (the science of finding weaknesses in crypto systems). Examples of the latter include Charles Babbage's Crimean War era work on mathematical cryptanalysis of polyalphabetic ciphers, redeveloped and published somewhat later by the Prussian Friedrich Kasiski. Understanding of cryptography at this time typically consisted of hard-won rules of thumb; see, for example, Auguste Kerckhoffs' cryptographic writings in the latter 19th century. Edgar Allan Poe used systematic methods to solve ciphers in the 1840s. In particular he placed a notice of his abilities in the Philadelphia paper Alexander's Weekly (Express) Messenger, inviting submissions of ciphers, of which he proceeded to solve almost all. His success created a public stir for some months.[24] He later wrote an essay on methods of cryptography which proved useful as an introduction for novice British cryptanalysts attempting to break German codes and ciphers during World War I, and a famous story, The Gold-Bug, in which cryptanalysis was a prominent element.

Cryptography, and its misuse, were involved in the execution of Mata Hari and in Dreyfus' conviction and imprisonment, both in the early 20th century. Cryptographers were also involved in exposing the machinations which had led to the Dreyfus affair; Mata Hari, in contrast, was shot.

In World War I the Admiralty's Room 40 broke German naval codes and played an important role in several naval engagements during the war, notably in detecting major German sorties into the North Sea that led to the battles of Dogger Bank and Jutland as the British fleet was sent out to intercept them. However its most important contribution was probably in decrypting the Zimmermann Telegram, a cable from the German Foreign Office sent via Washington to its ambassador Heinrich von Eckardt in Mexico which played a major part in bringing the United States into the war.

In 1917, Gilbert Vernam proposed a teleprinter cipher in which a previously prepared key, kept on paper tape, is combined character by character with the plaintext message to produce the cyphertext. This led to the development of electromechanical devices as cipher machines, and to the only unbreakable cipher, the one time pad.

During the 1920s, Polish naval-officers assisted the Japanese military with code and cipher development.

Mathematical methods proliferated in the period prior to World War II (notably in William F. Friedman's application of statistical techniques to cryptanalysis and cipher development and in Marian Rejewski's initial break into the German Army's version of the Enigma system in 1932).}

\end{document}